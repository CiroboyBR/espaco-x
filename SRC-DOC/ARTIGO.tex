\documentclass[12pt,twoside,a4paper]{book}

% ---------------------------------------------------------------------------- %
% Pacotes 
\usepackage[T1]{fontenc}
\usepackage[brazil]{babel}
\usepackage[latin1]{inputenc}
\usepackage{graphicx}
\usepackage[pdftex]{graphicx}           % usamos arquivos pdf/png como figuras
\usepackage{setspace}                   % espa�amento flex�vel
\usepackage{indentfirst}                % indenta��o do primeiro par�grafo
\usepackage{makeidx}                    % �ndice remissivo
\usepackage[nottoc]{tocbibind}          % acrescentamos a bibliografia/indice/conteudo no Table of Contents
\usepackage{courier}                    % usa o Adobe Courier no lugar de Computer Modern Typewriter
\usepackage{type1cm}                    % fontes realmente escal�veis
\usepackage{listings}                   % para formatar c�digo-fonte (ex. em Java)
\usepackage{titletoc}
%\usepackage[bf,small,compact]{titlesec} % cabe�alhos dos t�tulos: menores e compactos
\usepackage[fixlanguage]{babelbib}
\usepackage[font=small,format=plain,labelfont=bf,up,textfont=it,up]{caption}
\usepackage[usenames,svgnames,dvipsnames]{xcolor}
\usepackage[a4paper,top=2.54cm,bottom=2.0cm,left=2.0cm,right=2.54cm]{geometry} % margens
%\usepackage[pdftex,plainpages=false,pdfpagelabels,pagebackref,colorlinks=true,citecolor=black,linkcolor=black,urlcolor=black,filecolor=black,bookmarksopen=true]{hyperref} % links em preto
\usepackage[pdftex,plainpages=false,pdfpagelabels,pagebackref,colorlinks=true,citecolor=DarkGreen,linkcolor=NavyBlue,urlcolor=DarkRed,filecolor=green,bookmarksopen=true]{hyperref} % links coloridos
\usepackage[all]{hypcap}                % soluciona o problema com o hyperref e capitulos
\usepackage[square,sort,nonamebreak,comma]{natbib}  % cita��o bibliogr�fica alpha (alpha-ime.bst)
\fontsize{60}{62}\usefont{OT1}{cmr}{m}{n}{\selectfont}

% ---------------------------------------------------------------------------- %
% Cabe�alhos similares ao TAOCP de Donald E. Knuth
\usepackage{fancyhdr}
\pagestyle{fancy}
\fancyhf{}
\renewcommand{\chaptermark}[1]{\markboth{\MakeUppercase{#1}}{}}
\renewcommand{\sectionmark}[1]{\markright{\MakeUppercase{#1}}{}}
\renewcommand{\headrulewidth}{0pt}

% ---------------------------------------------------------------------------- %
\graphicspath{{./figuras/}}             % caminho das figuras (recomend�vel)
\frenchspacing                          % arruma o espa�o: id est (i.e.) e exempli gratia (e.g.) 
\urlstyle{same}                         % URL com o mesmo estilo do texto e n�o mono-spaced
\makeindex                              % para o �ndice remissivo
\raggedbottom                           % para n�o permitir espa�os extra no texto
\fontsize{60}{62}\usefont{OT1}{cmr}{m}{n}{\selectfont}
\cleardoublepage
\normalsize

% ---------------------------------------------------------------------------- %
% Op��es de listing usados para o c�digo fonte
% Ref: http://en.wikibooks.org/wiki/LaTeX/Packages/Listings
\lstset{ %
language=Java,                  % choose the language of the code
basicstyle=\footnotesize,       % the size of the fonts that are used for the code
numbers=left,                   % where to put the line-numbers
numberstyle=\footnotesize,      % the size of the fonts that are used for the line-numbers
stepnumber=1,                   % the step between two line-numbers. If it's 1 each line will be numbered
numbersep=5pt,                  % how far the line-numbers are from the code
showspaces=false,               % show spaces adding particular underscores
showstringspaces=false,         % underline spaces within strings
showtabs=false,                 % show tabs within strings adding particular underscores
frame=single,	                % adds a frame around the code
framerule=0.6pt,
tabsize=2,	                    % sets default tabsize to 2 spaces
captionpos=b,                   % sets the caption-position to bottom
breaklines=true,                % sets automatic line breaking
breakatwhitespace=false,        % sets if automatic breaks should only happen at whitespace
escapeinside={\%*}{*)},         % if you want to add a comment within your code
backgroundcolor=\color[rgb]{1.0,1.0,1.0}, % choose the background color.
rulecolor=\color[rgb]{0.8,0.8,0.8},
extendedchars=true,
xleftmargin=10pt,
xrightmargin=10pt,
framexleftmargin=10pt,
framexrightmargin=10pt
}

% ---------------------------------------------------------------------------- %
% Corpo do texto
\begin{document}
\frontmatter 
% cabe�alho para as p�ginas das se��es anteriores ao cap�tulo 1 (frontmatter)
\fancyhead[RO]{{\footnotesize\rightmark}\hspace{2em}\thepage}
\setcounter{tocdepth}{2}
\fancyhead[LE]{\thepage\hspace{2em}\footnotesize{\leftmark}}
\fancyhead[RE,LO]{}
\fancyhead[RO]{{\footnotesize\rightmark}\hspace{2em}\thepage}

\onehalfspacing  % espa�amento

% ---------------------------------------------------------------------------- %
% CAPA
% Nota: O t�tulo para as teses/disserta��es do IME-USP devem caber em um 
% orif�cio de 10,7cm de largura x 6,0cm de altura que h� na capa fornecida pela SPG.
\thispagestyle{empty}
\begin{center}
    \vspace*{2.3cm}
    \textbf{\Large{Espa\c{c}o - X}}\\
    
    \vspace*{1.2cm}
    \Large{Christian Sticchi, Emerson Barros e Oct\'{a}vio Magela}
    
    \vskip 4cm
    \textsc{
    Artigo apresentado a discplina de Projeto Interativo\\[-0.25cm]
    Centro Universit\'{a}rio Senac\\[-0.25cm]
   }
    
    \vskip 2cm
    Orientador: Professor Dr. Eduardo Heredia\\


   	\vskip 5cm
    \normalsize{S\~{a}o Paulo, dezembro de 2012}
\end{center}

% ---------------------------------------------------------------------------- %
\newpage

% Resumo
\textbf{{Resumo}}
\vskip 0.5cm
O Espa\c{c}o - X \'{e} um jogo 2D de espa\c{c}onaves. A estrat\'{e}gia \'{e} pensar r\'{a}pido. Ap\'{o}s surgir uma equa\c{c}\~{a}o na tela, a miss\~{a}o do jogador \'{e} destruir a nave inimiga de acordo com a resposta correspondente.
%% ------------------------------------------------------------------------- %%

\vskip 1cm
% Introdução
\textbf{{Introdu\c{c}\~{a}o}}
\vskip 0.5cm
Os inimigos estar\~{a}o dispostos em dez naves. As equa\c{c}\~{o}es s\~{a}o as chaves para ganhar pontos. Quanto mais r\'{a}pido o jogador definir o resultado, mais rapidamente conhecer\'{a} a mave desprotegida.


% ---------------------------------------------------------------------------- %
\vskip 1cm
% Desenvolvimento
\textbf{{Desenvolvimento}}
\vskip 0.5cm

O Espa\c{c}o - X foi desenvolvido por Christian Sticchi, Emerson Barros e Oct\'{a}vio Magela em um per\'{\i}odo de 4 meses. A etapa prim\'{a}ria do projeto foi definir a ideia central do jogo e a divis\~{a}o inicial de tarefas, onde cada integrante ficou encarregado por criar um escopo para o jogo. Feito isso, reunimos as ideias que eram comum a todos e definimos o que seria feito a partir daquele momento.

\vskip 0.5cm
No m\^{e}s seguinte, iniciamos o processo da escolha de arquivos (imagens, cores, sons, etc) e o desenvolvimento de cada fun\c{c}\~{a}o presente no jogo (atirar, mover, carregar imagens,etc).

\vskip 0.5cm
Finalmente o jogo estava praticamente pronto e nos restava apenas tirar bugs e corrigir os par\^{a}metros de algumas fun\c{c}\~{o}es.

% ---------------------------------------------------------------------------- %
\vskip 1cm
% Metodologia
\textbf{{Metodologia}}
\vskip 0.5cm

O Espa\c{c}o - X foi desenvolvido na linguagem C, com IDE Dev-C 4.9.9.2 e biblioteca gr\'{a}fica Allegro 4.2.1;

\vskip 1cm
% Conclusão
\textbf{{Conclus\~{a}o}}
\vskip 0.5cm
A implementa\c{c}\~{a}o do c\'{a}lculo de equa\c{c}\~{o}es do primeiro grau em um ambiente gr\'{a}fico sob o modelo de jogo 2D permite a evas\~{a}o do paradigma do aprendizado comum.
Essa fixa\c{c}\~{a}o de conhecimento ocorre atrav\'{e}s do est\'{\i}mulo subjetivo ao racioc\'{\i}nio l\'{o}gico (mesmo que seja de forma simples), ou seja, a miss\~{a}o destacada ao jogador \'{e} destruir naves, mas paralelo a isso \'{e} preciso conhecer o valor resposta da equa\c{c}\~{a}o para atingir o objetivo central do jogo.
% ---------------------------------------------------------------------------- %
\vskip 1cm
% Referências Bibliográficas
\textbf{{Refer\^{e}ncias Bibliogr\'{a}ficas}}
\vskip 0.5cm

<http://alleg.sourceforge.net/api.html>. API Allegro, 26 Nov 2006;
\vskip 0.5cm
<http://equipe.nce.ufrj.br/adriano/c/apostila/allegro/docs/allegro.html>.
A Biblioteca Gr\'{a}fica Allegro. Mota, Tiago, 2002.

% ---------------------------------------------------------------------------- %

\end{document}
